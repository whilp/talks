% vim: set tw=0 linebreak:
\documentclass{beamer}
\usepackage{graphicx}
\usepackage{booktabs}
\usepackage{hyperref}
\hypersetup{pdfborder={0 0 0 0}}

% Reasonable themes:
% Antibes Bergen Berkeley Berlin Frankfurt Goettingen Ilmenau Luebeck Malmoe
% Montpellier PaloAlto Rochester Singapore Szeged Warsaw bars boxes
% compatibility default lined plain shadow sidebar split tree
% And these ones include the author's name on every slide:
% Berkeley

% Declare themes.
\mode<presentation>
\usetheme{UWHEP}

% Personal macros.
\newcommand{\email}[1]{{\texttt #1}}
\newcommand{\newframe}[1]{\section{#1}
    \frametitle{\sc{#1}}}
\newcommand{\subframe}[1]{\subsection{#1}
    \frametitle{\sc{#1}}}
\newcommand{\supers}[1]{\ensuremath{^\textrm{#1}}}
\newcommand{\subs}[1]{\ensuremath{_\textrm{#1}}}
\newcommand{\ca}{\ensuremath{\sim}}

% Author information.
\title{Site report: Wisconsin}
\author[Maier]{
    Sridhara Dasu \\
    Dan Bradley, Ajit Mohapatra, Will Maier
    {\tt \{dasu,dan,ajit,wcmaier\}@hep.wisc.edu}}
\institute[Wisconsin]{University of Wisconsin - High Energy Physics}
\date[2009.03.03]{USCMS T2 Workshop - LIGO Livingston, 2009.03.03}
\logo{\includegraphics[height=0.6cm]{../../../Graphics/USCMS_logo.png}\hspace{.1cm}\includegraphics[height=0.75cm]{../../../Graphics/UW_logo.png}}

\begin{document}

% - Hardware deployments for FY09: what is the plan for meeting this year's
% targets, and what are the estimated costs, including costs for doing
% necessary replacements of older hardware?
% -- Are there facility issues at the site that we need to be aware of for
% longer-term planning of T2 capabilities?
% -- Some sites do better than others on the various metrics (RSV, SAM, job
% robot, etc.), but why?  What does your site do that helps you meet targets
% on metrics, and for targets you don't meet, why not?  Would more operations
% support help you?

\begin{frame}
    \titlepage
\end{frame}

\section{Overview}
\begin{frame}
    \tableofcontents
\end{frame}

\section{2008-2009 Facilities Status}
\subsection{Software}
\begin{frame}
\begin{table}
\begin{tabular}{lr}
    \toprule
    Component   &   Software \\
    \midrule
    OS          &   Scientific Linux 4.4 \\
    Batch       &   Condor 7.0.5 \\
    Storage     &   dCache 1.9.0-8 \\
    Grid        &   VDT 1.10.1r \\
    \bottomrule
\end{tabular}
\caption{2009 Software Status}
\label{2009_software_status}
\end{table}

\begin{itemize}
    \item OSG updates have become much more reliable, though we still can't relocate our installations
    \item Significant performance improvements (especially scaling) in recent Condor
    \item dCache running well, but replication is still a major problem
\end{itemize}

\end{frame}

\subsection{Batch}
\begin{frame}
% XXX: Use the following to generate N slots:
% condor_status -pool glow.cs.wisc.edu \
%   -constraint 'regexp("g[4-7]n\d+\.hep\.wisc\.edu", Machine)' \
%   -format '%s\n' Machine | wc -l
\begin{table}
\begin{tabular}{lrr}
    \toprule
    CPU Class               &   KSI2K   &   Slots \\
    \midrule
    2 x 2.80 GHz Xeon       &   210     &   110 \\  % g4-g7
    2 x 3.00 GHz Xeon       &   10      &   10 \\   % g8
    4 x 1.80 GHz Opteron    &   170     &   240 \\  % g9-g10
    8 x 2.66 GHz Xeon       &   180     &   120 \\  % g12
    8 x 3.00 GHz Xeon       &   250     &   260 \\  % g14
    \midrule
    Dedicated               &   820     &   740 \\
    Opportunistic           &   -       &   2460 \\
    \bottomrule
\end{tabular}
\caption{2009 Batch Status}
\label{2009_batch_status}
\end{table}

Since last workshop:
\begin{itemize}
    \item Added 32 8 x 3.00 GHz Xeon nodes
    \item Decommissioned 32 2 x 2.4 GHz Xeon nodes
\end{itemize}
\end{frame}

\begin{frame}
\begin{figure}
    % http://noc.hep.wisc.edu/nrg/condor/pools/GLOW-condor-claimed.1yr.gif
    \includegraphics[width=\textwidth]{Graphics/GLOW-condor-claimed-1yr.png}
    \caption{Dedicated and opportunistic slots, 2008-2009}
\end{figure}
\end{frame}

\subsection{Storage}
\begin{frame}
\begin{table}
\begin{tabular}{lr}
    \toprule
    Dedicated       &   280 TB \\   % s5, s15 (10 * 24 TB)
    Dual-purpose    &   200 TB \\   % rest
    \midrule
    Raw             &   480 TB \\
    Replicated      &   240 TB \\
    \bottomrule
\end{tabular}
\caption{2009 Storage Status}
\label{2009_storage_status}
\end{table}

Since last workshop:
\begin{itemize}
    \item Added \ca{}300 TB in dedicated and dual-purpose machines
    \item Switched from dCache ReplicaManager to PFM
\end{itemize}
\end{frame}

\begin{frame}
\begin{figure}
    % XXX: hacked from makeplot.py in nod's (broken) ganglia webui stuff
    \includegraphics[width=\textwidth]{Graphics/dcache-usage-1yr.png}
    \caption{dCache usage, 2008-2009}
\end{figure}
\end{frame}

\subsection{Network}
\begin{frame}
\begin{figure}
    % XXX: http://stats.net.wisc.edu/cgi-bin/genstatspage.pl?db=r-csscplat-b380-3-core_vl369_bytes&time=1y
    \includegraphics[width=\textwidth]{Graphics/network-1yr.png}
    \caption{WAN usage, 2008-2009}
\end{figure}

\begin{itemize}
    \item Tier2 traffic accounts for the majority of egress traffic from Wisconsin
    \item Ingress peaks due to PhEDEx transfers
\end{itemize}
\end{frame}

\section{2009 Deployment Plans}
\subsection{Software}
\begin{frame}
\begin{table}
\begin{tabular}{lp{.5\textwidth}}
    \toprule
    March-April     &   Complete Scientific Linux 5.2 testing and begin deployment \\
    \midrule
    May-June        &   Test BeStMan, dCache'? \\
    \midrule
    July            &   Upgrade to dCache feature branch, Chimera \\
                    &   \hspace{1cm}\emph{OR} \\
                    &   Begin switch to dCache' \\
    \bottomrule
\end{tabular}
\caption{2009 Software Deployment Plan}
\label{2009_software_deployment_plan}
\end{table}

\end{frame}

\subsection{Batch and storage systems}
\begin{frame}
\begin{table}
\begin{tabular}{lrr}
    \toprule
    Month           &   Storage (TB)    &   Slots \\
    \midrule
    March           &   100             &   200 \\
    June            &   300             &   - \\
    September       &   100             &   200 \\
    December        &   300             &   - \\
    \midrule
    New in 2009     &   800             &   400 \\
    Total in 2009   &   1280            &   1200 \\
    \bottomrule
\end{tabular}
\caption{2009 Hardware Deployment Plan}
\label{2009_hardware deployment_plan}
\end{table}

\begin{itemize}
    \item \$100kUSD budget for each round of alternating purchases:
    \begin{itemize}
        \item Dedicated: 8 x 2.66 GHz Xeon, 16 GB RAM, 24 x 1TB disk, LSI RAID
        \item Dual-purpose: 8 x 3 GHz Xeon, 16GB RAM, 4 x 1TB disk
    \end{itemize}
\end{itemize}

\end{frame}

\subsection{Network}
\begin{frame}
\begin{figure}
    % XXX: http://stats.net.wisc.edu/borderbytes.html
    \includegraphics[width=0.75\textwidth]{Graphics/network-all-2yr.png}
    \caption{Wisconsin campus border traffic, 2007-2009}
\end{figure}

\begin{itemize}
    \item Dynamically provision extra fiber (up to 20 Gb/s) dedicated to Tier2
    \item Teach PhEDEx agents to request new fiber before large transfers?
\end{itemize}
\end{frame}

\section{Facilities Planning}
\subsection{Cooling and Power}
\begin{frame}
\begin{figure}
    % XXX: http://noc.hep.wisc.edu/nrg/temp/MachineRoom2-temp.cgi
    \includegraphics[width=0.75\textwidth]{Graphics/mr2-temp-1yr.png}
    \caption{Machine room temperature (F), 2008-2009}
\end{figure}

\begin{itemize}
    \item At cooling capacity in new room; at power capacity in old room
    \item New room with shared space and power upgrade in old room forthcoming
\end{itemize}
\end{frame}

\section{How we try to do it}
\subsection{Emphasize local analysis}
\begin{frame}
\begin{figure}
    % XXX: http://www.hep.wisc.edu/cms/comp/operations/resourceusage.html
    \includegraphics[width=0.33\textwidth]{Graphics/condor-users.png}
    \caption{CPU hours by local username, 2005-2009}
\end{figure}

\begin{itemize}
    \item 140 local users, \ca{}50 running now
    \item Developed local scripts for quick analysis with central registration (Dan Bradley)
    \item Treat global production as if it were local (Ajit Mohapatra)
\end{itemize}
\end{frame}

\subsection{Redundancy}
\begin{frame}
\begin{itemize}
    \item Aggressive replication of all data (PFM) on small pools and commodity hardware
    \item Many GridFTP doors
    \item Dedicated central servers for dCache and GUMS (with hot spares)
    \item Multiple CEs with easy fail over
    \item Monitor the monitors: Nagios, central monitoring (JobRobot, SAM, RSV, et al)
\end{itemize}
\end{frame}

\end{document}
