% vim: set tw=0:
\documentclass{beamer}
\usepackage{graphicx}

% Reasonable themes:
% Antibes Bergen Berkeley Berlin Frankfurt Goettingen Ilmenau Luebeck Malmoe
% Montpellier PaloAlto Rochester Singapore Szeged Warsaw bars boxes
% compatibility default lined plain shadow sidebar split tree
% And these ones include the author's name on every slide:
% Berkeley

% Declare themes.
\mode<presentation>
\usetheme{UWHEP}

% Personal macros.
\newcommand{\email}[1]{{\texttt #1}}
\newcommand{\newframe}[1]{\section{#1}
    \frametitle{\sc{#1}}}
\newcommand{\subframe}[1]{\subsection{#1}
    \frametitle{\sc{#1}}}
\newcommand{\supers}[1]{\ensuremath{^\textrm{#1}}}
\newcommand{\subs}[1]{\ensuremath{_\textrm{#1}}}

% Author information.
\title{dCache Performance and Management}
\author[Will Maier]{Will Maier \\ \texttt{wcmaier@hep.wisc.edu}}
\institute[Wisconsin]{University of Wisconsin - High Energy Physics}
\date[8 March, 2007]{USCMS T2 Workshop, 8 March, 2007}
\logo{\includegraphics*[height=0.6cm]{../Graphics/USCMS_logo.png}\hspace{.1cm}\includegraphics*[height=0.75cm]{../Graphics/UW_logo.png}}

\begin{document}

\begin{frame}
    \titlepage
\end{frame}

\section{Overview}
\subsection{A Pareto Problem}

\begin{frame}
\frametitle{The "Trivial Many"\dots{}}
\begin{columns}[t]
\begin{column}{.25\textwidth}
    \begin{figure}
        \includegraphics*[width=.8\textwidth]{Graphics/Pareto.jpg}
        % \caption{The 80/20 guy.}
    \end{figure}
\end{column}
\begin{column}{.75\textwidth}
    \large{Most aspects of dCache function very, very well:}
    \begin{itemize}
        \item Most data is replicated efficiently and quickly
        \item Most transfers occur on a reasonable timescale
        \item Developers promise drool-inducing features (Chimera, PNFS 4.1; 
        see Patrick's talk)
        \item Many bugs have been fixed upstream
        \item More aspects of the system have been exposed in the database
        \item Strong community and patient experts resolve site problems
    \end{itemize}
    For site administrators, most of dCache requires little or no fiddling.
\end{column}
\end{columns}
\end{frame}

\begin{frame}
\frametitle{\dots{}and the "Vital Few"}
The remaining bits of dCache pose vital problems for site administrators.

\begin{itemize}
    \item What are these problems like?
    \begin{itemize}
        \item Monitoring and Accounting
        \item Reliability
        \item Community
    \end{itemize}
    \item Who is working on these problems?
    \begin{itemize}
        \item The busy dCache developers?
        \item The busy site admins?
        \item Someone else (who's probably busy, too)?
    \end{itemize}
    \item What can we do to improve this situation at our T2s?
\end{itemize}
\end{frame}

\begin{frame}
% \setcounter{tocdepth}{1}
\tableofcontents
\end{frame}

\section{Monitoring and Accounting}
\subsection{Pool Availability}
\begin{frame}
\frametitle{Pool Availability}
\begin{itemize}
    \item Status is listed on the 'cellInfo' page of the HTTP monitor
    \begin{itemize}
        \item \dots{}but web scraping is dumb
    \end{itemize}
    \item The admin interface can be scripted with {\tt sh(1)}
    \begin{itemize}
        \item \dots{}but this can be tedious and slow
    \end{itemize}
    \item The {\tt replicas} database has a {\tt pools} table
    \begin{itemize}
        \item Most convenient method for querying availability?
    \end{itemize}
\end{itemize}
\end{frame}

\subsection{File Replication}
\begin{frame}
\frametitle{File Replication}
\begin{itemize}
    \item PNFS makes all files available in a neat, unified namespace
    \begin{itemize}
        \item \dots{}but you can't get replication information from it
        \item \dots{}and it's awfully expensive to read the magic {\tt .(use)(2)(\$ID)} files
        \item \dots{}but we do get access to metadata via {\tt stat(2)}
    \end{itemize}
    \item Via the admin interface, we can ask each pool for its file catalog
    \begin{itemize}
        \item \dots{}but this can be tedious and slow
    \end{itemize}
    \item The PNFS {\tt companion} database has a {\tt cacheinfo} table
    \begin{itemize}
        \item This is great, but we don't get any metadata information
    \end{itemize}
\end{itemize}
\end{frame}

\subsection{Log Processing}
\begin{frame}
\frametitle{Log Processing}
\begin{itemize}
    \item At what level do we get useful info (not backtraces)?
    \item At what level do we avoid multi-GB log files?
    \item How can we centralize our logging?
    \begin{itemize}
        \item Especially important at sites which spread the core services across multiple servers
    \end{itemize}
    \item How do we monitor our logs for important events?
    \begin{itemize}
        \item What {\emph are} the important events?
    \end{itemize}
\end{itemize}
\end{frame}

\subsection[Testing]{Full-system and Regression Tests}
\begin{frame}
\frametitle{Full-system and Regression Tests}
\begin{itemize}
    \item How do we check for configuration or upgrade regressions?
    \item Test for:
    \begin{itemize}
        \item Performance
        \item Sanity
    \end{itemize}
    \item How do we test the full storage stack?
    \begin{itemize}
        \item Replication
        \item Pool-to-pool transfers
        \item Site-to-site transfers
    \end{itemize}
\end{itemize}
\end{frame}

\section{Reliability}
\subsection{cron-kicking}
\begin{frame}
\frametitle{cron-kicking}
\begin{itemize}
    \item Some elements require periodic kickage to function properly
    \begin{itemize}
        \item Pools?
        \begin{itemize}
            \item Nightly restarts via cron?
            \item Regular re-registration of files (via the admin interface)?
        \end{itemize}
        \item Doors?
        \item Core services?
        \begin{itemize}
            \item ReplicaManager?
        \end{itemize}
    \end{itemize}
\end{itemize}
\end{frame}

\section{Community}
\subsection{Documentation and Coordination}
\begin{frame}
\frametitle{Documentation and Coordination}
\begin{itemize}
    \item dCache.org provides The Book and other docs
    \item But {\emph we} don't centrally document our dCache experiences
    \begin{itemize}
        \item Some of our problems are specific to our project and dataflows
        \item dCache.org can't document every bizarre quirk we've stumbled across
    \end{itemize}
    \item A wiki would allow us to:
    \begin{itemize}
        \item Share previously undocumented solutions to common problems
        \item Document known bugs
        \item Coordinate effort between the sites
    \end{itemize}
\end{itemize}

% insert picture.

\end{frame}

\begin{frame}
\frametitle{A common scenario}
\begin{itemize}
    \item[1] Admin\ensuremath{_A} discovers Bug\ensuremath{_M}
    \item[2] Admin\ensuremath{_A} curses Bug\ensuremath{_M}
    \item[3] Admin\ensuremath{_A} resolves Bug\ensuremath{_M}
    \item[4] Admin\ensuremath{_A} tells the world about Bug\ensuremath{_M}
    \item[5] Admin\ensuremath{_A} learns that Admins\ensuremath{_{[B-H]}} have already discovered, cursed and resolved Bug\ensuremath{_M}
    \item[6] Admin\ensuremath{_A} discovers Bug\ensuremath{_{M+1}}\dots{}
    \item[7] Admin\ensuremath{_{A+1}} discovers Bug\ensuremath{_M}\dots{}
\end{itemize}

% Insert diagram.

\end{frame}

\end{document}

% Patrick's talk:
% Create a VDT/OSG -> dcache liason (GOC? -> VDT Ticket system @ UW)
% Use CMS-T2 list? (~VOs)
% FAQ?
% Neha's talk:
% dcache test suite? wtf? @ vdt.cs.wisc.edu?
