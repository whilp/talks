% vim: set tw=0:
\documentclass{beamer}
\usepackage{graphicx}

% Reasonable themes:
% Antibes Bergen Berkeley Berlin Frankfurt Goettingen Ilmenau Luebeck Malmoe
% Montpellier PaloAlto Rochester Singapore Szeged Warsaw bars boxes
% compatibility default lined plain shadow sidebar split tree
% And these ones include the author's name on every slide:
% Berkeley

% Declare themes.
\mode<presentation>
\usetheme{UWHEP}

% Personal macros.
\newcommand{\email}[1]{{\texttt #1}}
\newcommand{\newframe}[1]{\section{#1}
    \frametitle{\sc{#1}}}
\newcommand{\subframe}[1]{\subsection{#1}
    \frametitle{\sc{#1}}}

% Author information.
\title{dCache Performance and Management}
\author[Will Maier]{Will Maier \\ \texttt{wcmaier@hep.wisc.edu}}
\institute[Wisconsin]{University of Wisconsin - High Energy Physics}
\date[8 March, 2007]{USCMS T2 Workshop, 8 March, 2007}
\logo{\includegraphics*[height=0.6cm]{../Graphics/USCMS_logo.png}\hspace{.1cm}\includegraphics*[height=0.75cm]{../Graphics/UW_logo.png}}

\begin{document}

\begin{frame}
    \titlepage
\end{frame}

\section{Overview}
\begin{frame}
    \tableofcontents
\end{frame}

\subsection{dCache at Wisconsin}
\begin{frame}
\frametitle{dCache at Wisconsin}
\end{frame}

\subsection[80/20?]{An 80/20 Problem?}
\begin{frame}
\frametitle{80/20? 90/10?}
\end{frame}

\section{Monitoring}
\subsection{Pool Availability}
\begin{frame}
\frametitle{Pool Availability}
\end{frame}

\subsection{File Replication}
\begin{frame}
\frametitle{File Replication}
\end{frame}

\subsection{Log Processing}
\begin{frame}
\frametitle{Log Processing}
\end{frame}

\subsection[Testing]{Full-system and Regression Tests}
\begin{frame}
\frametitle{Full-system and Regression Tests}
\end{frame}

\section{Reliability}
\subsection{cron-kicking}
\begin{frame}
\frametitle{cron-kicking}
\end{frame}

\subsection{PNFS Registration}
\begin{frame}
\frametitle{PNFS Registration}
\end{frame}

\section{Community/Coordination}
\subsection{Problem Tracking}
\begin{frame}
\frametitle{Problem Tracking}
\end{frame}

\subsection{Wiki}
\begin{frame}
\frametitle{Wiki}
\end{frame}

\end{document}
